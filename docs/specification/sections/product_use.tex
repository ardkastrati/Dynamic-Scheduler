\section{Product use}

\subsection{Scope of application}
The dynamic scheduler is for scientist and researcher at "SimLab for Elementary- and Astro- Particle" (SCC). The scientist or researcher can use the dynamic scheduler to run the scientific code with different scheduling algorithm. The dynamic scheduler can also collect statistics, do bookkeeping and visualize the statistic and bookkeepings.


\subsection{Target group}

The target group are scientist and researcher at "SimLab for Elementary- and Astro- Particle" (SCC). The dynamic scheduler is called via the command line. The user has to have basic skills of using the command line and working on the supercomputer with "Moab Workload Manager". Moreover the scientist or researcher has to have basic skills in programming and compiling c++ with Message Passing Interface (MPI) to connect the scientific code with the dynamic scheduler.
\linebreak
(Nice-to-have: The dynamic scheduler provides a graphical user interface that allow running a new simulation on the supercomputer)


\subsection{Operation condition}

The following requirements must be met:
\begin{itemize}
	\item A parallel computer environment (supercomputer)
	\item A long enough time slice to finish the job on the supercomputer
	\item A installation of the Message Passing Interface (MPI)
	\item A scientific code that provides the following interface:
		\begin{itemize}
			\item code\_preprocessing\_master(run arguments): set of initial tasks
			\item code\_preprocessing\_slave(run arguments): void
			\item  code\_postprocessing\_master(void):void
			\item  code\_postprocessing\_slave(void):void
			\item run\_task(task):run result
		\end{itemize}
\end{itemize}