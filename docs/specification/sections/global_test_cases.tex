{
\newcounter{test}
\setcounter{test}{10}
\renewcommand{\labelitemi}{
	\ifnum \value{test}<10$/T 0\arabic{test} /$\addtocounter{test}{10}
	\else $/T \arabic{test} /$\addtocounter{test}{10}\fi
}

\section{Global test cases}
	\subsection{Test cases}
		\subsubsection{Scheduler tests}
			\begin{itemize}
				
				%TEST
				\item Start the scheduler managed with a single-queue master-worker design 				
				
					\begin{enumerate}
						\item Description\newline
In this test case the scheduler should be successfully started via MOAB Workload Manager with the "single-queue" option in the command line.
						\item Related requirements\newline
 							  /F10/		
					\end{enumerate}					  
											
				
				
				\item Start the scheduler managed with a multiple-queue task-stealing design
				\begin{enumerate}
						\item Description\newline
In this test case the scheduler should be successfully started via MOAB Workload Manager with the "multiple-queue" option in the command line.
						\item Related requirements\newline
 							  /F20/		
				\end{enumerate}	
				
				
				%TEST
				\item Select a scheduling strategy 
			    \begin{enumerate}
						\item Description\newline
							Test whether the scheduler is able to select the following strategies
							\begin{itemize}
							 	\item First in first out (FIFO)
							 	\item Last in first out (LIFO)
							 	\item Shortest job first (SJF)
							 	\item Longest job first
(LJF)
								
							\end{itemize}
							
						\item Related requirements\newline
 							  /F30/, /F40/, /F50/, /F60/		
				\end{enumerate}	
				
				
				%TODO implement
				%\item Change the strategy during the scheduling process
				%\begin{enumerate}
				%		\item Description\newline
							%Test whether the scheduler is able to change the current 									strategy during the program, if the current strategy is 									inefficient.
				%		\item Related requirements\newline
 				%			  /F70/		
				%\end{enumerate}				
				
				\item Add a new strategy to program
				\begin{enumerate}
					\item Description\newline
Test whether it is possible for the user to add strategies to the program by using a strategy interface.
					\item Related requirements\newline
						/F80/
				\end{enumerate}
					
				
				%TEST
				\item Process Command Line Interface (CLI) parameter 		
					\begin{enumerate}
						\item Description\newline
Test whether the scheduler is able to process a command line interface parameter.
						\item Related requirements\newline
 							  /F90/		
				\end{enumerate}	
				
				%TEST
				\item Initialize the MPI World 
				\begin{enumerate}
						\item Description\newline
Test whether the scheduler successfully creates and initializes the MPI World with the given arguments after it starts.
						\item Related requirements\newline
 							  /F100/		
				\end{enumerate}



				%TEST
				\item Execute master functions
				\begin{enumerate}
						\item Description\newline
Test whether the scheduler, which is managed with a single-queue master-worker design, is able to execute the following master functions: 
					\begin{enumerate}
\item code\_preprocessing\_master(:RunArguments)
						\item code\_postprocessing\_master(:RunArguments)
					\end{enumerate}
						\item Related requirements\newline
 							  /F110/	, /F130/	
				\end{enumerate}
				
				
				
				%TEST
				\item Execute worker functions
				\begin{enumerate}
						\item Description\newline
Test whether the workers are able to execute the following worker functions: 
					\begin{enumerate}
						\item code\_preprocessing\_slave(:RunArguments)
						\item code\_postprocessing\_slave(:RunArguments)
					\end{enumerate}
						\item Related requirements\newline
 							  /F120/	, /F130/
				\end{enumerate}
				
				
				%TEST
				\item Receive new tasks from scientific code
				\begin{enumerate}
					\item Description\newline
Test whether the scheduler is able to receive new tasks from the scientific code and place them correctly in the queue.
					\item Related requirements\newline
						/F140/
				\end{enumerate}
				
				
				
				%TEST
				\item Query the Data Mining module, how much resources a specific task needs
				\begin{enumerate}
					\item Description\newline
Test whether the scheduler is able to query the Data Mining module to estimate the weight of the task.
					\item Related requirements\newline
						/F170/
				\end{enumerate}
				
				
				%TEST
				\item Read data from the bookkeeping database
				\begin{enumerate}
					\item Description\newline
Test whether the scheduler is able to correctly and trouble-free read the data from the bookkeeping database
					\item Related requirements\newline
						/F180/
				\end{enumerate}
				
				
				%TEST
				\item Write data into the bookkeeping database
				\begin{enumerate}
					\item Description\newline
Test whether the scheduler is able to correctly and trouble-freely write the data into the bookkeeping database
					\item Related requirements\newline
						/F190/
				\end{enumerate}
				
				
				%TEST 
				\item Send a specific number of tasks to available workers
				\begin{enumerate}
					\item Description\newline
Test whether the scheduler is able to correctly send different tasks to available workers using the current strategy
					\item Related requirements\newline
						/F200/
				\end{enumerate}
								
			
			
			
			%TEST 
				\item Steal a task from another queues
				\begin{enumerate}
					\item Description\newline
Test whether a scheduler with an empty queue is able to steal tasks from other schedulers
					\item Related requirements\newline
						/F390/
				\end{enumerate}
							
			
			
			%TEST 
				\item Use an already created statistical file
				\begin{enumerate}
					\item Description\newline
Test whether the scheduler is able to use an already created statistical file.
					\item Related requirements\newline
						/F400/
				\end{enumerate}
								
			\end{itemize}
		
		\subsubsection{Database tests}
			\begin{itemize}
				
		
				%TEST 
				\item Initialize database
				\begin{enumerate}
					\item Description\newline
Test whether the scheduler successfully creates and initializes the database.
					\item Related requirements\newline
					/F230/
				\end{enumerate}
			\end{itemize}
		
		\subsubsection{Data mining tests}
			\begin{itemize}
			
			%TEST 
				\item Read data from statistics database
				\begin{enumerate}
					\item Description\newline
Test whether data mining module is able to correctly and trouble-freely read data from statistics database
					\item Related requirements\newline
					/F240/
				\end{enumerate}
				
			%TEST
				%\item Write data into statistics database
				%\begin{enumerate}
				%	\item Description\newline
%Test whether data mining module is able to correctly and trouble-freely write data from %statistics database
%					\item Related requirements\newline
%					/F250/
%				\end{enumerate}
				
				
				
				%TEST 
				\item Analyse the statistics to define resource requirements for a task
				\begin{enumerate}
					\item Description\newline
Test whether the data mining module is able to approximate the resource requirements for a task and improve its results as time passes and more data is stored into the database.
					\item Related requirements\newline
					/260/, /270/
				\end{enumerate}
				
				
				
				%TEST 
				\item Store runtime information after a specific task is finished
				\begin{enumerate}
					\item Description\newline
Test whether the data mining module is able to store relevant runtime information of completed tasks.
					\item Related requirements\newline
					/F280/
				\end{enumerate}
				
			
		\end{itemize}	
		\subsubsection{Bookkeeping tests}
		\begin{itemize}
		
		%TEST
		\item Bookkeep data for a specific task
				\begin{enumerate}
					\item Description\newline
Test whether is possible to bookkeep the following data for a specific task: 	
					\begin{enumerate}
					\item Task appears
					\item Task started
					\item Task finished 
					\item Interconnection between processes
					\end{enumerate}
					
					\item Related requirements\newline
					/F290/
				\end{enumerate}
		\end{itemize}
		\subsubsection{Statistics tests}
		\begin{itemize}
		
		
		
		%TEST 
				\item Read and write data into statistics file.
				\begin{enumerate}
					\item Description\newline
					\begin{enumerate}
					
					\item 
Test whether the program is able to correctly read from the file.
					\item
Test whether the program is able to correctly write new task informations into the statistics file.
					\end{enumerate}
					\item Related requirements\newline
					/F320/
					
				\end{enumerate}
				
				
				%TEST 
				%\item Search in statistics
				%\begin{enumerate}
				%	\item Description\newline
%Test whether the program is able to efficiently search information from the statistics file. 
%					\item Related requirements\newline
%					/F330/
%				\end{enumerate}
				
				
				%TEST 
%				\item Duplicate files on too high amount of requests
%				\begin{enumerate}
%					\item Description\newline
%Test whether the program is able to duplicate files when there are too many requests for the same file.
%					\item Related requirements\newline
%					/F340/
%				\end{enumerate}
				
				
				
				%TEST 
%				\item Delete duplicated files 
%				\begin{enumerate}
%					\item Description\newline
%Test whether the program is able to delete duplicated files when there are not many tasks left or the scientific code is finished.
%					\item Related requirements\newline
%					/F350/
%				\end{enumerate}
				
				
				
				%TEST 
				\item Interpret information of the statistic file
				\begin{enumerate}
					\item Description\newline
Test whether the program is able to read and correctly interpret the information from the statistics file, in order to approximate the runtime of a specific task.
					\item Related requirements\newline
					/F360/
				\end{enumerate}
				
		\end{itemize}		
%		\subsubsection{Visualisation tests}%TODO nice graph
%		\begin{itemize}
		
		%TEST 
%				\item Show graph in the graphical user interface
%				\begin{enumerate}
%					\item Description\newline
%Test whether the program correctly shows the graph with the current progress of the program and periodically updates it
%					\item Related requirements\newline
%					/F410/,/F420/,/F430/
%				\end{enumerate}
				
		
		%TEST 
%				\item Export bookkeeping data to .vtk
%				\begin{enumerate}
%					\item Description\newline
%Test whether the program correctly exports the bookkeeping data to .vtk format
%					\item Related requirements\newline
%					/F390/
%				\end{enumerate}
				
		
		%TEST 
%				\item Export statistics data to .vtk
%				\begin{enumerate}
%					\item Description\newline
%Test whether the program correctly exports the statistics data to .vtk format
%					\item Related requirements\newline
%					/F400/
%				\end{enumerate}
				
%		\end{itemize}				
		
%	\susubbsection
		
		\subsection{Test scenarios}
		
		\subsubsection{Basic test scenarios}
		
		\begin{enumerate}
			\item Perform a simple scheduling managed with a single-queue master-worker design with a dummy code with a specific number of tasks.
			\begin{enumerate}
				\item /T10/, /T60/ Start the scheduler with a single-queue master-worker design
				\item /T70/ Initialize the MPI World
				\item /T30/ Select a default strategy
				\item /T80/ Execute master function (code\_preprocessing\_master)
				\item /T140/ Send a task to available workers
				\item /T90/ Execute worker functions (code\_preprocessing\_slave)
				\item /T100/ Receive new tasks from scientific code
				\item /T110/ Query Data Mining module, how muc resources the current task needs
				\item /T240/ Read data from the statistics file
				\item /T200/ Analyse data to define resource requirements for the task
				\item /T80/ Execute master function (code\_postprocessing\_master)
				\item /T80/ Execute worker function (code\_postprocessing\_master)
				\item /230/ Bookkeep data after the task is finished
				%\item /T290/, /T300/ Export bookkeeping and statistic data to .vkt file
			\end{enumerate}
			
			\item Perform a simple scheduling managed with a multiple-queue task-stealing algorithm
				\begin{enumerate}
				
				\item /T20/, /T60/ Start the scheduler with the task-stealing design
				\item /T70/ Initialize the MPI World
				\item /T30/ Select a default strategy
				\item /T80/ Execute master function (code\_preprocessing\_master)
				\item /T140/ Send a task to available workers
				\item /T90/ Execute worker functions (code\_preprocessing\_slave)
				\item /T100/ Receive new tasks from scientific code
				\item /T110/ Query Data Mining module, how much resources the current task needs
				\item /T240/ Read data from the statistics file
				\item /T200/ Analyse data to define resource requirements for the task
				\item /T150/ Steal a task from another queue
				\item /T80/ Execute master function (code\_postprocessing\_master)
				\item /T80/ Execute worker function (code\_postprocessing\_master)
				\item /230/ Bookkeep data after the task is finished
				%\item /T290/, /T300/ Export bookkeeping and statistic data to .vkt file
				
				\end{enumerate}
			
			%\item Perform a dynamic scheduling with a (dummy) scientific code, that produces tasks, which can't be efficiently processed with the current strategy 
				
			%	\begin{enumerate}
			%		\item /T10/, /T60/ Start the scheduler with a single-queue master-worker design
			%	\item /T70/ Initialize the MPI World
			%	\item /T30/ Select a default strategy
			%	\item /T80/ Execute master function (code\_preprocessing\_master)
			%	\item /T140/ Send a task to available workers
			%	\item /T90/ Execute worker functions (code\_preprocessing\_slave)
			%	\item /T100/ Receive new tasks from scientific code
			%	\item /T40/ Change the current strategy during the scheduling process
			%	\item /T110/ Query Data Mining module, how muc resources the current task needs
			%	\item /T240/ Read data from the statistics file
			%	\item /T200/ Analyse data to define resource requirements for the task
			%	\item /T80/ Execute master function (code\_postprocessing\_master)
			%	\item /T80/ Execute worker function (code\_postprocessing\_master)
			%	\item /230/ Bookkeep data after the task is finished
			%	\item /T290/, /T300/ Export bookkeeping and statistic data to .vkt file
			%	\end{enumerate}
			
			
			\item Add new strategies to program
			
			\begin{enumerate}
				\item /T50/ Add a new strategy to program
				\item Then use the first or second test scenario
			\end{enumerate}
			
			
			\item Run the same scientific code twice
				\begin{enumerate}
					\item Use the first or second test scenario
					\item /T160/ Use an already created statistic file to initialize the database initialization.
				\end{enumerate}
		\end{enumerate}
		
		
		     
		   \subsubsection{Stress testing scenarios}
		   
		   \begin{enumerate}
		   	% \item Perform a scheduling with a code, which produces a lot of tasks that take very little time to complete
		   	 %  \begin{enumerate}
		   	  % \item /T260/ Duplicate files on too high amount of request
		   		%	\item Use first or second basic test scenario 
		   		%	\item /T270/ Delete duplacated files
		   			
		   	%   \end{enumerate}
		   	\item Perform a scheduling with a code, which produces a few of tasks that take a lot of time to complete
		   	
		   	   \begin{enumerate}
		   			\item Use the first or second basic test scenario
		   	   \end{enumerate}
		   	\item Perform a scheduling with a code, where the completion time of tasks differs a lot from task to task. Some tasks take a lot of time to complete compared to the other ones.
		      	\begin{enumerate}
		      	%	\item /T260/ Duplicate files on too high amount of request
		   			\item Use the third basic test scenario  in order to dynamically change the strategy
		   		%	\item /T270/ Delete duplacated files
		      	\end{enumerate}
		   \end{enumerate}	
}