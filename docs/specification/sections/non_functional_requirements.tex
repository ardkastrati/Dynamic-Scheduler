{
\newcounter{funcNFR}
\setcounter{funcNFR}{10}
\renewcommand{\labelitemi}{
	\ifnum \value{funcNFR}<10$/NF 0\arabic{funcNFR} /$\addtocounter{funcNFR}{10}
	\else $/NF \arabic{funcNFR} /$\addtocounter{funcNFR}{10}\fi
}

\section{Non-functional requirements}
	\begin{itemize}
		\item The dynamic scheduler is paralleled and uses the Massage Passing Interface (MPI) for communication between different CPU's
		\item The dynamic scheduler runs in a 'MPI World' up to 3.000.000 CPU's
		\item The dynamic scheduler has minimized overhead. The focus should be on running the scientific simulation on the CPU and not the scheduler
		\item The dynamic is possible to connect the scientific code with the scheduler with basic C++ skills
		\item Command line interface of the dynamic scheduler has to be intuitive
		\item The dynamic scheduler does not manipulate scientific tasks
		\item The dynamic scheduler does not manipulate scientific code
		\item The dynamic scheduler does not affect the result of the scientific simulations
		\item It is possible to integrate new scheduling strategies easily
		\item It is easy to continue the development of the dynamic scheduler
	\end{itemize}
}