{
\newcounter{funcNFR}
\setcounter{funcNFR}{10}
\renewcommand{\labelitemi}{
	\ifnum \value{funcNFR}<10$/NF 0\arabic{funcNFR} /$\addtocounter{funcNFR}{10}
	\else $/NF \arabic{funcNFR} /$\addtocounter{funcNFR}{10}\fi
}

\section{Non-functional requirements}
	\begin{itemize}
		\item The dynamic scheduler is paralleled and uses the Message Passing Interface (MPI) for communication between different CPUs
		\item The dynamic scheduler runs in a 'MPI World' up to 30.000 CPUs
		\item The dynamic scheduler has minimized overhead. The focus should be on running the scientific simulation on the CPU and not the scheduler
		\item The user has the possibility to connect the scientific code with the scheduler with basic C\texttt{++} skills%TODO spelling
		\item The command line interface (CLI) of the dynamic scheduler has to be intuitive
		\item The dynamic scheduler does not manipulate scientific tasks
		\item The dynamic scheduler does not manipulate scientific code
		\item The dynamic scheduler does not affect the result of the scientific simulations
		\item It is possible to integrate new scheduling strategies using the scheduling strategies interface
		\item The dynamic scheduler is easy to maintain and extend
	\end{itemize}
}